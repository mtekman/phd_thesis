
% No "CHAPTER" headers
\usepackage{titlesec}
\titleformat{\chapter}[block]
  {\normalfont\huge\bfseries}{\thechapter.}{1em}{\Huge}
\titlespacing*{\chapter}{0pt}{-19pt}{0pt}

% Bibliotecque
\usepackage[square,sort,comma,numbers]{natbib}
\usepackage{bibentry}
\bibliographystyle{IEEEtran}


% Global footnote numbering
\usepackage{chngcntr}
\counterwithout{footnote}{chapter}



\newcommand{\ignore}[1]{}
\newcommand{\nobibentry}[1]{{\let\nocite\ignore\bibentry{#1}}}
%% apsrev entries in the text need definitions of these commands
%\newcommand{\bibfnamefont}[1]{#1}
%\newcommand{\bibnamefont}[1]{#1}



%Numbering for subsubsub
\setcounter{secnumdepth}{4}
\renewcommand{\thesubsubsection}{\roman{subsubsection}}

\usepackage{setspace} % set line spacing within a \begin\end environment

\usepackage[font=small,labelfont=bf]{caption}
\usepackage{graphicx} % support the \includegraphics command and options
\usepackage{paralist} % very flexible & customisable lists (eg. enumerate/itemize, etc.)
\usepackage{verbatim} % adds environment for commenting out blocks of text & for better verbatim
\usepackage{subfig} % make it possible to include more than one captioned figure/table in a single float
\usepackage[hyphens]{url}
\usepackage{enumerate}

\usepackage{booktabs} % Better looking tables
\usepackage{tablefootnote}

\usepackage{cleveref}  %cites figures intelligently
\usepackage{float}  %These two ensure that table position follows text by specifying {table}[H]
\usepackage{wrapfig} % Figure wrapping

\usepackage{amssymb}

%% Graphics
% Images here
\newenvironment{figurehere}
  {\def\@captype{figure}}
  {}
\makeatother

%Arrows in tablesTT
\usepackage{tikz}
\usetikzlibrary{matrix}


\newcommand*\circled[3]{\tikz[baseline=(char.base)]{
            \node[shape=circle,draw=#3,#2] (char) {#1};}}

\newcommand*\circleSolid[1]{\circled{#1}{solid}{blue}}
\newcommand*\circleDashed[1]{\circled{#1}{dashed}{blue}}


%CODE LISTINGS
\usepackage{color,colortbl}
\definecolor{RED}{rgb}{1,0,0}
\definecolor{ORANGE}{rgb}{1,0.5,0}
\definecolor{YELLOW}{rgb}{1,1,0}
\definecolor{LGREEN}{rgb}{0.25,1,0}
\definecolor{GREEN}{rgb}{0,1,0}
\definecolor{SGREEN}{rgb}{0,1,0.5}
\definecolor{CYAN}{rgb}{0,1,1}
\definecolor{TEAL}{rgb}{0,0.5,1}
\definecolor{BLUE}{rgb}{0,0,1}
\definecolor{INDIGO}{rgb}{0.5,0,1}
\definecolor{PURPLE}{rgb}{1,0,1}
\definecolor{MAGENTA}{rgb}{1,0,0.5}



\usepackage{listings}
\usepackage{caption}

% consistent listing spacing
\usepackage{enumitem}
\usepackage{lipsum}


\definecolor{lightgray}{rgb}{.9,.9,.9}
\definecolor{darkgray}{rgb}{.4,.4,.4}
\definecolor{purple}{rgb}{0.65, 0.12, 0.82}
\lstdefinelanguage{JavaScript}{
  keywords={break, case, catch, continue, debugger, default, delete, do, else, false, finally, for, function, if, in, instanceof, new, null, return, switch, this, throw, true, try, typeof, var, void, while, with, SKIP, FOR, IN, RANGE, IF, UNTIL, EXISTS, NOT, BREAK, ELSE, WHILE, RETURN, FUNCTION},
  morecomment=[l]{//},
  morecomment=[s]{/*}{*/},
  morestring=[b]',
  morestring=[b]",
  ndkeywords={class, export, boolean, throw, implements, import, this},
  keywordstyle=\color{blue}\bfseries,
  ndkeywordstyle=\color{darkgray}\bfseries,
  identifierstyle=\color{black},
  commentstyle=\color{purple}\ttfamily,
  stringstyle=\color{red}\ttfamily,
  sensitive=true
}

\lstset{
	tabsize=4,
%	language=matlab,
        	basicstyle=\scriptsize,
%     	upquote=true,
%       	aboveskip={\baselineskip},
        	columns=fixed,
        	showstringspaces=false,
        	extendedchars=true,
        	breaklines=true,
	prebreak = \raisebox{0ex}[0ex][0ex]{\ensuremath{\hookleftarrow}},
	frame=single,
        	showtabs=false,
        	showspaces=false,
        	showstringspaces=false,
        	identifierstyle=\ttfamily,
        	keywordstyle=\color[rgb]{0,0,1},
        	commentstyle=\color[rgb]{0.133,0.545,0.133},
        	stringstyle=\color[rgb]{0.627,0.126,0.941},
	language=JavaScript
}

\usepackage{footnote}
%\makesavenoteenv{description}

\usepackage{cancel} % i.e. cancel to zero in math mode

%No indents
%\setlength{\parindent}{15pt}

%%% PAGE DIMENSIONS
\usepackage[a4paper, total={5.5in,8in}]{geometry}
\usepackage[T1]{fontenc} % set input encoding (not needed with XeLaTeX)


%Verbatim in footnote
\newsavebox\myVerb
\newenvironment{verbbox}{\lrbox\myVerb}{\endlrbox}
\newcommand*{\verbBox}{\usebox\myVerb}



