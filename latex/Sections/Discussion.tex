\chapter{Discussion}

\section{Pipeline Evaluation}

Small analyses such as the 3-bit and 5-bit pedigrees examined in the previous chapter are generally not very informative for linkage by themselves. For the 3-bit \gls{bio:pedigree} we observed multiple low \gls{bio:LOD} score flat peaks that did not exclude many regions, and for the 5-bit pedigree we saw fewer flat peaks with slightly higher LOD scores and still not that many regions of exclusion.

That is not to say small pedigrees are unusable; weak studies can lend strength to larger analyses where the linkage region is one of a multitude of peaks which can be narrowed down by the extra regions of exclusion given by a smaller analysis. Similarly, large analyses are of great assistance to smaller ones, where we witnessed a 29-bit pedigree lend power to the haplotype analysis of two 3-bit pedigrees in Figure\pref{fig:res:29bitpedother}.

Pedigree errors such as those indicated by GRR, gender, or Mendelian errors are typically not hindered by the size of analysis\footnote{except for GRR runtimes, where window polling increases with the number of total genotypes.} and problems are detected at the \gls{bio:trio} level. Often, small pedigrees are later expanded into larger ones by genotyping the rest of the family in order to improve the LOD score. By catching these small trio errors early on, a researcher can make a more informed decision on whether to pursue the family for further linkage. As we saw in the pedigree in Figure\pref{fig:res:7bitgenotypes}, parents sometimes have to falsify their children's genealogy for personal and legal reasons, and this can be cause for many a headache for the researcher who has to reconcile the true pedigree.

Medium size analyses (7-bit to 18-bit) are usually in the vicinity where such scenarios occur because the linkage peaks are reasonably high ($>$ 1) and few in number ($<$ 6), meaning that there is more room for a researcher to "play" with analysis by running more variations upon the input sets to maximize the analysis. Again, it should be stated that maximizing an analysis by small modifications to the pedigree and penetrance model are not necessarily a true representation of the phenotype, but only what is expected or allowed according to the classical ideals of genetics (e.g. parent-offspring genotype variance and Mendelian inheritance).

For large analyses ($\geq$ 19-bit) fewer analyses are possible due to time constraints, where Table\pref{table:res:allegrosinglechroms} shows that the timeframe shifts from a under an hour ($\sim$ 2800 seconds $\sim 47$ minutes) to a few days ($\sim$ 945000 seconds $\sim 11$ days). 

Hardware limitations (namely RAM) also make Allegro largely unuseable on a genome-wide scale because the over 19-bit modification switches off the \gls{comp:CUDD} capability which would otherwise minimize the number of repeated calculations employed in the linkage. It is better suited to analysing individual chromosomes, preferably higher numbered \gls{bio:chromosomes} since the lower chromosomes begin to scale exponentially with runtime as shown in Figure\pref{fig:res:singleplots}. Simwalk is of more use on the genome-wide level, but risks not being able to accurately determine X-linked analyses.


\subsection{Pipeline Summary}

The filtering portion of the pipeline seems to work extremely well for all types of pedigrees and penetrances, and is the main strength of the pipeline. The combination of Mendelian errors, GRR relations, gender validation, and unlikely genotypes form easily understandable plots of the quality of the data. Errors are pinpointed to specific families and specific individual(-trios), and the quick runtime of the filtering core ($<$ 5 minutes) means that changes to the pedigree can be tested and evaluated very easily. 

In relation to the rest of the pipeline, analyses are typically brute-force driven in order to maximize the LOD score as permitted by the reasonable runtimes for all pedigrees under the 19-bit limit. Over the limit requires more thought before an analysis is run, often requiring a pedigree to be split into smaller families\footnote{especially for consanguinous pedigrees.} to form a smaller-bit combined analysis that may provide more visual cues on which specific chromosomes to explore when running the larger pedigree, rather than risk the time cost of running a full \gls{bio:genome} analysis via Simwalk\footnote{Or Allegro, if the pedigree is X-linked and time is not a crucial issue.}.


\section{HaploHTML5 Evaluation}

\subsection{HaploBlock Resolution}
As shown in Figure~\ref{fig:res:haplo29compare} and Figure~\ref{fig:res:haplo23compare} in Section\prefstar{ref:res:hapcomp} HaploHTML5 resolves \gls{bio:autosomal} haplotypes just as well as HaploPainter, with the added benefit of comparing haplotypes of any selected individuals in a side-by-side view.

The 15-bit X-linked pedigree shown in Figures \{\ref{fig:res:haplo20comparetop},~\ref{fig:res:haplo20comparemid},~\ref{fig:res:haplo20comparebot}\} on pages~\pageref*{fig:res:haplo20comparetop} to ~\pageref*{fig:res:haplo20comparebot} shows how the chromosome X limitation inherent in HaploPainter is overcome in HaploHTML5, where the highly improbably blocks shown in HaploPainter are resolved correctly using the appropriate penetrance model.

\subsection{UI Operability}
The PedCreate view is intuitive and simple, with two types of relations of possible (mate-mate, and parent-offspring) with valid connections being highlighted to the user via red or white anchor points\footnote{See Figure\pref{fig:haplo:pedcreate} as an example.}. Individuals are trivially added, modified, deleted, and can be moved around with their various relationship lines updating their positions (and their siblings and/or mates lines) accordingly. An invidiual's name can also be specified, but it is not a requirement. The resultant pedigree can be exported to pedfile, with optional positional meta tags which can be reread back into the application either by manually selecting the exported file, or by saving the pedigree into local storage and reloading it upon startup.

The SelectionView mode as shown in Figure\pref{fig:haplo:selectmode} allows for multi-family selection which simplifies the comparison process, especially when comparing the disease locus (found via linkage) within combined analyses. The \gls{comp:DOS}\label{ref:disc:dos} view outlined in page~\pageref{ref:haplo:dos} is in intriguing concept that aims to simplify generational representation upon the selected indivduals under analysis, but ultimately complicates it. It may be better to simply represent the pedigrees of all selected individuals via a small \gls{comp:mipmap} thumbnail in the corner of the screen.

The Homology tools mode is a useful tool for scoring regions of sequence homology under heterozygous, homozygous, and compound \gls{bio:heterozygous} models. These scores are plotted vertically against the marker scale (see Figure\pref{fig:haplo:homomode}) for quick scrolling, and can also be output to a text file for later inspection. However, the mode can be confusing to interpret due to the non-binary nature in which it scores homology (see page~\pageref{ref:haplo:homologyscoredet}) and the modified HaploPainterRFH script is better in highlighting regions of homology as shown in Figure\pref{fig:res:haplo29comparehomology}. It would be more beneficial for the end-user if the homology tools merely outlined the regions that met that maximum homology score for the scenario (as in the case of HaploPainterRFH) instead of trying to overlay all the data at once.

\subsection{Performance}

Performance testing of the A* best-first algorithm upon a single set of test data gave negligible runtimes on Desktop (0.041 s) and Laptop (0.302 s). Mobile (1.523 s) produced a minor noticeable delay, but still produced the correct haplotypes, which was unexpected given the lightweight nature of its Javascript engine (EmbedLite) not necessarily incorporating all the features of ECMCAScript6 specification. The overall differences seem to be a factor of 10 from Mobile $\rightarrow$ Laptop $\rightarrow$ Desktop, but the differences between \gls{comp:OS} and browser upon the same hardware appears to be function specific as expected \cite{ratanaworabhan2010jsmeter}, with minor speed increases in Chromium over Firefox.

HaploPainter runs on \gls{comp:Perl} and relies upon \gls{comp:Tk}, \gls{comp:Cairo}, \gls{comp:Sort Naturally}, and \gls{comp:DBI} Perl dependencies to be present in the system. The more dependencies a program has, the less likely it is to be ported to other platforms successfully since it's dependencies must be ported first. Perl itself also is not very common on \gls{comp:ARM} devices\footnote{The Jolla mobile device used in the Results chapter runs an OS aimed at \gls{comp:Linux} and shell enthusiasts; Perl did exist upon it, however none of the Perl dependencies had been ported to it and so HaploPainter could not run on it.}. However, the only dependency that HaploHTML5 requires is a web-browser with Javascript capability, immediately making it much more portable than HaploPainter.

\subsection{Distribution}

HaploHTML5 was always considered to be distributed open source


\section{Future Work}

The pipeline and application are in a useable state, and both will achieve the tasks they are designed for: performing quality linkage on large data, and rendering haplotypes in-browser. However, there are a few improvements that could be incorporated to make them more intuitive to use at the end-user level, and also adjustments that could be made at the core level to increase performance.

\subsection{Pipeline improvements}

\subsection{Fixing Installer Issues}

The pipeline comes packaged with an installer that is designed to work on all Unix and Linux platforms; and so far it has been successfully upon Ubuntu, Debian, Arch, and Gentoo. The installation is not straightforward however, since every OS does not always use the same package manager as another. \gls{comp:Ubuntu} and \gls{comp:Debian} use Aptitute (or simply \textit{apt}), \gls{comp:Arch} uses \textit{pacman}, and \gls{comp:Gentoo} uses \textit{portage}.

Their install syntax is very similar, but they are all separate package installers and must be treated as such. To ease this issue, all software packages are kept in a separate file which merely lists them so that they can be fed into each OS's respective package manager. Unfortunately, though there is generally good accordance of package names between different Linux OS's, there can be some minor discrepancies in their naming that can cause issues\footnote{The switchover from Python2 to Python3 for example creates some very ambiguous instances where a package named \textit{python-gobl} is not inherently clear on which version of \gls{comp:Python} the gobl library is for.}.

Another issue is installing the Perl dependencies (for HaploPainter). If HaploPainter will be depreciated in the future with the uptake of HaploHTML5, then this issue can be easily resolved by simply removing the Perl dependencies from the installer altogether. Otherwise, the problem arises when deciding which means of software distribution to install the Perl dependencies from: the OS package manager, or via \gls{prog:CPAN} \cite{cpan} which installs Perl modules to user or root directories separate from the system.

The latter is the more compatible option, since CPAN exists on every system that supports Perl, but the packages from CPAN tend to be newer than those distributed by the system repositories and this may cause clashes.


\subsection{Custom Modular Component Ordering}

The pipeline as outlined in Figure\pref{fig:methpipe} exists as two main modules: Filtering (comprising of three sub-modules), and Linkage, with pre-linkage and post-linkage operations.

Quick retesting usually involves modifying the main pipeline script to some degree in order to prematurely terminate the analysis after the Filtering stage. This is trivial for any technician familiar with \gls{comp:Bash} or Unix who can comment out the relevant portions, but may prove more difficult for those used to graphical interfaces.

A better method would be use a configuration text file that outlines what modules and submodules the pipeline should run before terminating. The default configuration would run all modules in their specific orders, but a custom user configuration could change the order of the modules and better tailor the analysis to their needs (e.g. maximising a pedigree requires only the Filtering module).

\subsection{Automate Large Pedigree Handling}

Pedigrees beyond the 19-bit limit are not handled by the default pipeline. Only the pre-linkage and Filtering components are handled and then the pipeline has to terminate in order for the custom scripts that can handle the large pedigrees can be called manually by the user.

This can be easily automated, since the \textit{bit\_score.py} script automatically determines the bit-size of any pedigree, and can act as a screening stage for whether the pipeline should run the vanilla Allegro binary, or the custom modified (over 19-bit) Allegro binary. A potential pitfall would be having to declare beforehand whether the analysis should be haplotypes or multi-point parametric driven, since the modified binary cannot do both, but this can be specified in a configuration text file mentioned in the previous subsection.

\subsection{Fixing Allegro}

The modified Allegro binary works by deactivating the \gls{comp:CUDD} library which minimizes the number of unique computations made in the LOD score calculations and vastly improves upon memory and disk resources. Allegro uses version 2.4.1 of the CUDD library, but a later version exists that may fix the problem.

Attempts to incorporate this library into Allegro, but certain function calls that Allegro depended upon were depreciated between versions and remapping them to newer ones proved problematic. 

One other solution to this problem would be to rewrite Allegro from scratch. Allegro is written in C/C++ and is well commented in the source code. A more up-to-date rendition of the same program, using some of the newer features of the C++11 specification \cite{c++2011iso} may greatly benefit the program and even boost performance under the previous C++98 specification under which it was written and compiled.

\subsection{GRR Backend}

The most problematic component of the pipeline is the graphical GRR interface, which is a Windows binary that does not take any commandline arguments\footnote{Despite a pleading email to the developers who promised to add such functionality in the future, only to then respond with silence.}. As a result data cannot be simply fed into the application from the commandline where it performs all computations in the background, instead the application requires a display server to host its graphical interface where user-clicks are then simulated by the system in order to load in the data.

The requirements of GRR are immense; entire display libraries must be installed such as a X11 or Wayland display server, a desktop environment, display fonts, sounds, and all manner of other multimedia dependencies that come packaged with operating systems that require desktop sessions to run. The application also requires \gls{prog:wine} to run because it was originally written for Windows.

A replacement Linux-based binary would be ideal at this stage, and prompting our own rewrite of GRR may be required.


\subsection{Web UI}

Each run comes equipped with an automatic readme generator so that each run has a log of what analysis transpired, from what input sources, at what date, with how many markers, and under what penetrance. The penetrance and number of \gls{bio:markers} are automatically detected from the surrounding input files, but everything else must be typed by the user.

A web front-end would facilitate in this greatly, where the user would simply upload their files to the webserver which would place them in a correctly named folder (see page~\pageref{ref:meth:foldconv}) as specified by a project ID input field within the form, filter for informative markers as set by other options in the form, and would print the details into the readme file without the user having to manually specify anything.

The web interface could also give options for customizing the order of pipeline modules and write an appropriate configuration script.

This could be implemented in pure \gls{comp:PHP} \cite{bakken2000php} which can talk to the system and is capable of spawning off bash commands and reporting on their feedback in realtime. The images generated during the Filtering stage (pedigrees, Mendelian errors, GRR) could be displayed within the interface as and when they appear.

The linkage portion of the pipeline would have to be reported to the user via commandline output which would be fed into a textarea in the interface, but once complete the linkage plots could be displayed as tiled images or opened up as separate PDFs in a new tab.

The user would then be able to download their data as a compressed ZIP file as packaged using the normal \gls{prog:collect.sh} script from a button or link that would only become visible once the analysis is complete.

One issue that has been known to arise when implementing similar interfaces in other pipelines\footnote{Significant work has also been performed upon a High-Throughput \textit{Sequence} Analysis Pipeline.} is that the pipeline is spawned by the browser which makes the browser the parent process. Once the browser is closed (or a \gls{comp:SIGHUP} signal is detected) the system will attempt to terminate all child processes too which may kill an ongoing process.

To get around this, the \gls{prog:screen} utility is used to detach the child process from the parent but still maintain control later on. However getting realtime output out of a screen process is not straightforward, and requires taking repeated "snapshots" of the session window and reading the contents.


\subsection{Deployment}

There are three ways that the pipeline can be deployed:
\begin{enumerate}
\item{{\bf Install} - The pipeline is installed onto a Unix or Linux system via the installer script to work alongside the other files and programs on the OS.}
\item{{\bf Live Environment} - the entire pipeline is placed into a bootable medium (DVD or USB) and all analyses are run in-house off the medium. Live environments are restricted in the notion that the operating system is essentially read-only, which has the benefit of greatly reducing any \gls{comp:paging} overhead from the OS itself, but means that all created files must be kept in memory. For small analyses this is not a problem, but larger pedigrees will prove a problem.}
\item{{\bf Web hosting} - the pipeline is hosted from a dedicated server with all optimizations catered towards the specific hardware it runs upon. Users then interface with the pipeline remotely either through the \gls{prog:ssh} utility via the secure \gls{comp:SSL} protocol, or through a \gls{comp:HTTP} server to a web interface. Complete analyses are transferred through \gls{prog:scp}, \gls{prog:rsync}, or downloaded from the web interface via HTTP.}
\end{enumerate}

The first and third option are the most viable, albeit with the install stage requiring the user to take care of their own hardware which may be a large limiting factor in the overall uptake of pipeline usage due to heavy hardware requirements. The web hosting option is more approachable, but requires limiting the number of concurrent users accessing the pipeline at any time since it can only handle one large pedigree at a time, though numerous small pedigrees can be handled at any given time.



\subsection{Effective Filesystem Management}

The working directory should be seen as a temporary file system at best, since many linkage programs generate a substantial amount of temporary files during processing. Though modern file systems are excellent at keeping track of and recovering files (a task known as \gls{comp:journaling}), it is still good practice to "mount a scratch monkey" such that temporary working files are isolated away from archived ones in distinctly separate partitions.

It is preferable to isolate the two file systems on different disks altogether, such that a disk-wide read error on a given hard disk won't affect the operation of another. Most modern file systems have journaling enabled by default, which writes extra data to disk so that a recovery is possible in the event of a crash, but this becomes more problematic with Solid State Drives (\gls{comp:SSD}s) which have a limited number of read/write operations, and are prone to seizing should an operating system over-breach those limits.

This is not an improbable occurrence; operating systems routinely write temporary files to disk in order to manage \gls{comp:RAM} constraints in a process known as \gls{comp:paging}, where not immediately required portions of system memory can be sequestered into slower disk storage to free up space in the RAM for more higher priority tasks.

When a system begins to run low on \gls{comp:RAM} (as in the frequent case of extensive linkage analysis), a great deal of paging is performed to keep the linkage program in memory, leading to a substantial amount of disk writes on a journaled file-system. Mechanical (spin) drives can handle these read/write requests robustly, but they are an order of magnitude slower than SSDs and significant bottlenecks can occur where the system has to wait for the disk to be ready in order to perform a block\footnote{If multiple sequential operations are requested on same contiguous portion of a disk, then the operating system groups the individual requests under a large contiguous 'block' request, that performs one long read/write operation instead of several short ones.}    read/write to it.


\subsubsection{Why Upgrading Disk Technology Wont Necessarily Lend a Performance Increase}
It is clear that SSDs offer no advantage at all in the use-case of frequent large temporary file activity, since though the operating system will rarely have to wait in order to perform a read/write, the limitations on the number of these operations shortens their lifetime considerably.

The best long-term compromise is to use mechanical disks with a filesystem without journaling; either an older filesystem where journaling was never implemented (though file size constraints may be enacted), or a newer filesystem where journaling is disabled upon initialization. This reduces the total number of read/write operations whilst still ensuring disk longevity.

Recoverability may be jeopardized, but the temporary nature of the files within the system imply that a re-run of the same analysis would not be resumable in any case; for it is the operating system that dispatches the temporary files to the linkage programs, and not the linkage program itself (with the  notable exception of \gls{prog:Simwalk}).


\subsubsection{RAID Configuration}

One extra caveat to reduce the number of operations upon a single disk \textit{and} keep some level of redundancy is to distribute the load across several disk drives in a standard redundant array of independent disks (\gls{comp:RAID}); specifically a RAID-5 level setup. RAID relies on the method of storing contiguous data across several mediums (a concept known as \gls{comp:striping}), as well as ensuring the data is correct and error-checkable without having to scan the entire block of data (a concept known as \gls{comp:parity}). The failure of one disk drive would not be enough to impede the setup and an ongoing analysis could be reconstituted from the other remaining disks, though a minimum of three identically sized disks is required.

These features lend well to a large temporary filesystem, since the storage capacities of multiple disks can now be combined to store a great amount of data, as well as the number of operations on any one disk being reduced by a factor of \textit{n} (for \textit{n} disks in the setup). The setup will still have the same read/write speeds as an ordinary disk, but there will not be any added complexity since the operating system will detect it as a single volume.


\subsection{HaploHTML5 Improvements}

Other than the general bugfixes, as well as rewriting the \gls{comp:DOS} method mentioned previously (page~\pageref{ref:disc:dos}), there are numerous improvements that could be made to the application in order to boost its popularity amongst geneticists.


\subsubsection{Founder Colour Group Representation}
One current problem with the application is the limited colour space for the founder allele groups to be allocated to. Under the \gls{comp:HSV} model with a hue scored from 0\% to 100\% differences in colour become harder to distinguish below the 15\% range as shown in Figure~\ref{fig:disc:huescale}. 

For $f$ founders, there are $2f$ unique colours that must be assigned to their respective alleles, meaning that as soon as the pedigree begins to have more than 3 founders\footnote{$ \frac{100}{15} \sim 7 $ unique colour groups for all founder \gls{bio:alleles} $ \sim 3$ founders.}, the colour groups become more ambiguous.

One solution would be to change the the Value component of the HSV model to take on darker and/or lighter components, which would double and/or triple the number of available colours, but would require changing the colour of the genotypes text which reside on top of the coloured block. 

\fignow{Sections/images/hue_scale.png}
{Hue Scale (\%). Distinct colours are more visible at 15\% increments.}
{fig:disc:huescale}
{0.7}{}

It is possible to move the genotype text out next to the block as represented in HaploPainter, but this would reduce the amount of available horizontal space since more width would have to allocated per individual.

Another solution would be to regenerate the founder allele colour groups to only those currently under selection, but introduces the issue of inconsistent colouring across multiple runs on the same data.


\subsubsection{Analysis Sharing and Privacy}

The application currently operates in-browser via either local or web deployment, and analyses are restricted to a single user. In the interests of scientific collaboration, it is likely that the end-user would want to share their analysis with another who may be working on the same project.

In order to achieve this, there are two main avenues of development to explore:
\begin{enumerate}
\item{Restrict application deployment to cloud-based services, where analyses are stored in a \gls{comp:mySQL}\footnote{The mariaDB open source fork to be specific, since mySQL was acquired by Oracle \cite{oraclemysql}.} database and are issued private login keys to restrict access to the data to only permitted users.

Access can then be metered using established social network authentication services such as Google, Facebook, Microsoft, Twitter, and OpenID via the \gls{comp:OAuth2} \cite{hardt2012oauth} protocol.

The issue of patient confidentiality would still remain an issue however, since shared analyses would store patient data remotely on the site host, and could be subject to vulnerabilities if the host is compromised.  To complicate matters, the PedCreate View allows for the option of applying actual names to individual IDs, which is not an issue if the application uses locally-stored data, but is a major concern if such names are stored in a remote server.

A solution to this would be to store two copies of the pedigree: one unmodified within local storage, and the other in remote storage with patient names stripped from the pedigree. The user could then share their analysis with another (with names omitted) over the internet, and still preserve names when working locally on their own machine. However, problems will quickly arise if the other user wishes to modify the pedigree, in which case the locally-stored names become moot, where it may then be beneficial to restrict modification altogether and only allow read-only sharing to solve this problem.

Another concern is that stripping names from the pedigrees may not be enough to preserve patient confidentiality, since the pedigree structure itself may be enough to identify a family, especially if the \gls{bio:phenotype} is rare.
}
\item{Analyses are encoded within the URL, where nothing is actually stored in the remote database. The pedigree is compressed via \gls{comp:LZMA} \cite{lzma} in order to keep the number of characters within the recommended 2000 character limit for URLs \cite{urllength} and encoded under \gls{comp:base64} where the string is then affixed to a \gls{comp:GET} specifier the URL to be shared.

Once a recipient clicks on the URL, the GET variable is decoded and uncompressed, where it is then loaded as a regular pedigree file on the recipient's browser. 

Base64 encoding is supported by the HTML5 specification \cite{masinter1998data}, and the lz-string library exists \cite{lzstring} to perform LZMA compression and decompression within Javascript.

Though data is never stored remotely, this approach does run the risk of anyone with the link being able to view the data.
}
\end{enumerate}

A combination of both approaches would likely need to be implemented to achieve both privacy and security.


\subsubsection{Framework Upgrade}

Eric Dowell is the author of \gls{comp:KineticJS}\footnote{the framework used extensively in the application's development.} but has since moved onto another project after he completed KineticJS. KineticJS was deemed "complete" in 2014 \cite{kinetictar} and has not been worked upon since. This was good news for developers who needed a stable platform to work with, but came with the unspoken agreement that applications developed under it would not be future maintained.

Dowell has since moved onto \gls{comp:ConcreteJS} \cite{concretejs} which is a more general-purpose 2D library from manipulating shapes and is vastly optimized such that the entire codebase (including documentation) consists of no more than 600 lines. The project is still in the early Alpha phase (v0.1.0), but looks very promising. It shares almost the same syntax as KineticJS and will be non-problematic to implement.

